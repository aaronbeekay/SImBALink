\documentclass{article}
% packages
\usepackage{gensymb}
\usepackage{amsmath}
\usepackage{graphicx}
\usepackage{verbatim}		% \comment{}

% smaller page margins
\usepackage[top=2in, bottom=1.5in, left=1.5in, right=2in]{geometry}

% inline TODO: notes
\usepackage[textwidth=1.75in]{todonotes}

% for \FloatBarrier
\usepackage{placeins}

% graphics setup
\DeclareGraphicsExtensions{.eps,.pdf,.png,.jpg}

\begin{document}
\section{Block diagram}
	\begin{figure}[h!]
			\centering
			\includegraphics[width=5in]{Figures/Battery_pack_block_diagram}
			\caption{Battery pack block diagram}
			\label{fig:battery_pack_equivalent_circuit}
	\end{figure}
	\FloatBarrier

\section{Inputs, outputs, and parameters}
	\subsection{Inputs}
	\begin{tabular}{ l | l | l  }
		Input					&	Symbol		&	Unit		\\	\hline
		DC current			&	$I_{dc}$		&	A		\\
		%Ambient temperature	&	$T_{amb}$	&	\degree C \\ 
	\end{tabular}
	
	\subsection{Outputs}
	\begin{tabular}{ l | l | l  }
		Output					&	Symbol		&	Unit		\\	\hline
		Internal charge				&	$Q$			&	coulomb	\\
		Terminal voltage			&	$V_{dc}$		&	V		\\
		Internal temperature			&	$T_{pack}$	&	\degree C	\\
	\end{tabular}
		
\section{Background, rationale, modeling strategy}
	\subsection{Electrical model}
		Each battery cell is modeled as an equivalent circuit:
		
		\begin{figure}[h!]
				\centering
				\includegraphics[width=4in]{Figures/Battery_pack_equivalent_circuit}
				\caption{Battery cell equivalent circuit}
				\label{fig:battery_pack_equivalent_circuit}
		\end{figure}
		\FloatBarrier
		
		where:
		\begin{center}
		\begin{tabular}{l l}
			$V_{oc}(SOC)$	&	is the battery open-circuit voltage in volts	\\
			$R_{0}$			&	is the battery ohmic resistance in ohms \\
			$R_{1}$			& 	is the battery first-order resistance in ohms \\
			$C_{1}$			& 	is the battery firs-order capacitance in farads
		\end{tabular}
		\end{center}
		
		The battery open-circuit voltage, $V_{oc}$, is a function of the battery state of charge ($\frac{Q}{Q_0}$), and is represented by a lookup table.
		
		A pack is modeled as $n$ series cells connected by $n-1$ resistive busbars, as shown in Figure \ref{fig:Battery_pack_series_model}.
		
		\begin{figure}[h!]
				\centering
				\includegraphics[width=3in]{Figures/Battery_pack_series_model}
				\caption{Battery pack series equivalent circuit model (with $n$ = 3)}
				\label{fig:Battery_pack_series_model}
		\end{figure}
		\FloatBarrier
		
		With the assumptions above, an equation can be written for the terminal voltage of each individual cell:
		
		\begin{equation}
			V(t) = V_\text{oc}(\text{SOC}) - I_\text{dc} R_0 - I_{dc}\left( R_1 \parallel Z_{C1} \right).
		\end{equation}
		\todo{Finish this in differential equation form}
		
	\subsection{Thermal model}
		The pack is currently modeled at a constant temperature.
	
		\begin{comment}
		%This is not currently implemented
		
		The battery pack is modeled as a single thermal mass which has some thermal resistance to ambient temperature:
		
		\begin{figure}[h!]
				\centering
				\includegraphics[width=\linewidth]{Figures/Battery_pack_thermal_equivalent_circuit}
				\caption{Battery cell thermal equivalent circuit}
				\label{fig:Battery_pack_thermal_equivalent_circuit}
		\end{figure}
		\FloatBarrier
		
		The waste power (heat input) $P_{waste}$ is the sum of the heat dissipated in each resistance:
		
		\begin{equation}
			P_{waste} = I_{dc} (R_{ohm})^2 + I_{Rct} (R_{ct})^2 + I_{Rdf} (R_{df})^2 + I_{sd} (R_{sd})^2
		\end{equation}
		
		and the thermal resistance to ambient, $\theta_{ma}$, is an arbitrary nonlinear function of vehicle speed, represented by a 1D lookup table:
		
		\begin{equation}
			\theta_{ma} = h(v)
		\end{equation}
		
		where $v$ is the vehicle's longitudinal velocity.
		\end{comment}
		
		

\section{Parameters}
	\subsection{Parameters}
	\begin{tabular}{ l | l | l | l }
		Parameter						&	Symbol		& MATLAB symbol	&	Unit		\\	\hline
		Initial stored charge					&	$Q_0$		& \texttt{Q\_0}		&	coulomb	\\
		Number of series cells				&	$n$			& \texttt{n}		&			\\
		Zero-order series resistance, per cell	&	$R_0$		& \texttt{R0}		&	ohm		\\
		First-order resistance, per cell			&	$R_1$		& \texttt{R1}		& 	ohm		\\
		First-order capacitance, per cell		&	$C_1$		& \texttt{C1}		&	farad	\\
		Open-circuit voltage					&	ocv(SOC)		& -				&	volt		\\
		Initial temperature					&	$T_0$		& \texttt{T\_0}		&	\degree C \\
		
	\end{tabular}
	
	\todo{There are more parameters.}

\end{document}